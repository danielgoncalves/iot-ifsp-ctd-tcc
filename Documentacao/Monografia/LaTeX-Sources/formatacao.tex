\usepackage[utf8]{inputenc}
\usepackage[T1]{fontenc}
\usepackage[english, main=brazil]{babel}
\usepackage{amsmath}
\usepackage{amssymb,amsfonts,textcomp}
\usepackage{color}
\usepackage{array}
\usepackage{supertabular}
\usepackage{listings}         % Para as linguagens de programação
\usepackage{lastpage}		  % Usado pela Ficha catalográfica
\usepackage{indentfirst}	  % Indenta o primeiro parágrafo de cada seção.
\usepackage{hhline}
\usepackage{hyperref}
\usepackage[pdftex]{graphicx}
\graphicspath{ {./figuras/} }


% retira as mensagens de aviso do pacote Glossaries
\let\printglossary\relax
\let\theglossary\relax
\let\endtheglossary\relax
% coloca Seções em maiúsculas sem negrito no Sumário
\makeatletter
\let\oldcontentsline\contentsline
\def\contentsline#1#2{%
  \expandafter\ifx\csname l@#1\endcsname\l@section
    \expandafter\@firstoftwo
  \else
    \expandafter\@secondoftwo
  \fi
  {%
    \oldcontentsline{#1}{\normalfont\MakeTextUppercase{#2}}%
  }{%
    \oldcontentsline{#1}{#2}%
  }%
}
\makeatother
% ---
% Pacotes glossaries
% ---
\usepackage[subentrycounter,seeautonumberlist,nonumberlist=true]{glossaries}
% para usar o xindy ao invés do makeindex:
%\usepackage[xindy={language=portuguese},subentrycounter,seeautonumberlist,nonumberlist=true]{glossaries}
% ---
% Citações de referências no formato alfabético e negrito
\usepackage[alf, abnt-emphasize=bf]{abntex2cite} 
% Margens definidas em 25 mm para uso como documento em PDF
% Para imprimir use as seguintes margens:
% \usepackage[left=30mm, top=30mm, right=20 mm, bottom=20mm] {geometry}

\usepackage[margin=25 mm]{geometry}
%---
% O arquivo com o nome dos alunos e dos orientadores é lido aqui.
% Atualizar diretamente no arquivo
%---

%%%%%%%%%%%%%%%%%%%%%%%%%%%%%%%%%%%%%%%%%%%%%%%%%%%%%%%%%%%
% Corrige a fonte dos capítulos, seções, resumos, etc.
%%%%%%%%%%%%%%%%%%%%%%%%%%%%%%%%%%%%%%%%%%%%%%%%%%%%%%%%%%%

\renewcommand{\ABNTEXchapterfont}{\bfseries \rmfamily}  % Capítulos em Bold e Maiúsculas
\renewcommand{\ABNTEXchapterfontsize}{\normalsize}
\renewcommand{\ABNTEXsectionfont}{\rmfamily}            %  Seções em  Maiúsculas apenas
\renewcommand{\ABNTEXsectionfontsize}{\normalsize}
\renewcommand{\ABNTEXsubsectionfont}{\bfseries}         % Subseções em Bold apenas
\renewcommand{\ABNTEXsubsectionfontsize}{\normalsize}
\renewcommand{\lstlistingname}{Código}                  % Nome para os códigos no texto
\renewcommand{\lstlistlistingname}{Lista de \lstlistingname s}
\makeatletter                                          % Configura a linha da lista de códigos
\renewcommand\l@lstlisting[2]{{\normalfont\@dottedtocline{1}{1.5em}{2em}{Código~#1}{#2}}}
\makeatother
\usepackage{url16023}  % para retirar < e > da URL nas referências.