\section{Trabalhos Correlatos}

\subsection{Taíra e Siqueira, 2018}

Daniel Taíra e Felipe Siqueira publicaram em 2018 um trabalho intitulado "PROTOTIPAGEM UTILIZANDO PLATAFORMA ARDUINO PARA SISTEMA DE CONTROLE DE NÍVEL" \cite{tairaprototipagem}. O projeto é desenvolvido de modo que os componentes (sensores) responsáveis pela medição, os componentes responsáveis pela movimentação do fluído (atuadores) e os módulos de controle, estão todos ligados em um sistema monolítico, capaz de fazer a movimentação do fluído de um tanque inferior para um tanque superior em um aparato onde todos os componentes, incluindo os tanques, sejam parte uma única peça.

\subsection{Matheus Gonçalves, 2019}

Matheus Gonçalves publicou em 2019 um trabalho intitulado "CONTROLE DE NÍVEL DE PLANTA DIDÁTICA USANDO CONTROLADOR LÓGICO PROGRAMÁVEL" \cite{gonccalves2019controle}. O projeto emprega o uso de um CLP didático em oposição à plataformas mais acessíveis como Arduíno ou ESP32. O propósito do projeto é controlar os níveis de fluídos entre três tanques, sendo que dois deles servem como reservatório de fluído para o tanque principal, que é o tanque monitorado medindo-se o nível através de um sensor ultrassônico HC-SR04. Os tanques reservatórios são mantidos dentro do mínimo e máximo através de sensores do tipo chave-boia. 

\subsection{Thalys Gadelha, 2020}

Thalys Gadelha publicou em 2020 um trabalho intitulado "SISTEMA PARA MONITORAMENTO DO NÍVEL DE ÁGUA EM RECIPIENTES DE ANIMAIS DOMÉSTICOS" \cite{gadelha2020sistema}. Trata-se de um projeto voltado para a manutenção de vasilhames para alimentação de animais domésticos. Utiliza ESP32 e sensores para determinar se o nível do conteúdo medido está abaixo de um nível considerado crítico, para então utilizar uma rede GSM para enviar notificações de SMS, além de um sinal sonoro (buzzer). Neste trabalho não há implementação de alimentação automática do conteúdo.

\subsection{Sumário da Correlação}

Embora os projetos mencionados tenham certas similaridades em relação a este projeto, todos eles apresentam a solução entre a medição, monitoramento e acionamento das bombas de escoamento ou alimentação em um bloco monolítico. Observa-se que os projetos citados estão muito mais focados na precisão da medição e do acionamento.

Este trabalho, por outro lado, preocupa-se mais com a infraestrutura e com a integração e comunicação entre os componentes, no sentido de que procura aplicar as características que distinguem a IoT, de modo a desenvolver uma plataforma para formação de uma rede de equipamentos capazes de medir e atuar de forma independente e com alto potencial de escalabilidade.
